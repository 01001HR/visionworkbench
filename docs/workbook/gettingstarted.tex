\chapter{Getting Started}\label{ch:gettingstarted}

This chapter describes how to get started with actually using the
Vision Workbench.  It describes how to obtain the Vision Workbench and
its prerequisite libraries, how to build and install it, and how to
build a simple example program.  This chapter does {\it not} discuss
how to program using the Vision Workbench.  If that's what you're
looking for then skip ahead to Chapter~\ref{ch:workingwithimages}.

\section{Obtaining the Vision Workbench}

Most likely if you are reading this document then you already know 
where to obtain a copy of the Vision Workbench sources, if you haven't 
obtained them already.  However, if not, a link to the most up-to-date 
distribution will always be available from the NASA Ames open-source 
software website, at \verb#opensource.arc.nasa.gov#.

In addition to obtaining the Vision Workbench, you will also need to 
obtain and install whatever pre-requisite libraries you will need.  
The only requirement is the Boost C++ Libraries, a set of extensions 
to the standard C++ libraries that is available at \verb#www.boost.org#.
These days many Linux systems come with some version of Boost already 
installed, generally in the directory \verb#/usr/include/boost#.  The 
Vision Workbench has been tested with Boost versions 1.32 and later.

\begin{table}[t]\begin{centering}
\begin{tabular}{|l|l|l|} \hline
Name    & Used By            & Source                                   \\ \hline \hline
Boost   & All                & \verb#http://www.boost.org/#              \\ \hline
PNG     & FileIO (opt.)      & \verb#http://www.libpng.org/#             \\ \hline
JPEG    & FileIO (opt.)      & \verb#http://www.ijg.org/#                \\ \hline
TIFF    & FileIO (opt.)      & \verb#http://www.libtiff.org/#            \\ \hline
OpenEXR & FileIO (opt.)      & \verb#http://www.openexr.com/#            \\ \hline
PROJ.4  & Cartography (req.) & \verb#http://www.remotesensing.org/proj/# \\ \hline
GDAL    & Cartography (opt.) & \verb#http://www.remotesensing.org/gdal/# \\ \hline
GLEW    & GPU (req.)         & \verb#http://glew.sourceforge.net/#       \\ \hline
CG      & GPU (opt.)         & \verb#http://developer.nvidia.com/#       \\ \hline
\end{tabular}
\caption{A summary of Vision Workbench dependences.}
\label{tbl:dependencies}
\end{centering}\end{table}

Other libraries are required only if you want to use particular
features of the Vision Workbench.  A summary of the various libraries
that the Vision Workbench will detect and use if present is given in
Table~\ref{tbl:dependencies}.  It lists the particular Vision
Workbench module that uses the library, whether it is required or
optional for that module, and where the library can be obtained.
Details of each of the modules and the features that are enabled by
each dependency are given in the corresponding sections of this book.
If you are just starting out with the Vision Workbench, it is
generally fine to begin only with Boost.  You can always go back and 
rebuild the Vision Workbench with support for additional features 
later if you discover that you need them.

\section{Building the Vision Workbench}

If you are using a UNIX-like platform such as Linux or Mac OS it is
generally straightforward to build the Vision Workbench once you have
installed any necessary libraries.  First unpack the distribution, go
to the distribution's root directory, and configure the build system
by running ``\verb#./configure#''.  This script will examine your machine
to determine what build tools to use and what libraries are installed
as well as where they are located.  Near the end of its output it will
list whether or not it was able to find each library and which Vision
Workbench modules it is going to build.  You should examine this
output to confirm that it was able to find all the libraries that you
had expected it to.  If not then you may need to configure the build
system to search in the right places, as discussed in
Section~\ref{sec:config-build}.

Assuming the output of the \verb#configure# script looks good, you can
now proceed to build the Vision Workbench itself by running 
``\verb#make#''.  Since most of the Vision Workbench is header-only,
``building'' the Vision Workbench like that should be relatively
quick.  Once that's done it's almost certainly a good idea to confirm
that things are working properly by building and running the unit
tests typing ``\verb#make check#''.  If there are no erros, the final
step is to install the Vision Workbench headers, library, and sample
programs using ``\verb#make install#''.  By default the installation
location is the directory \verb#/usr/local#, so you will probably need
to obtain the necessary priveleges to write to it using a command such
as \verb#su# or \verb#sudo#.  If you do not have administrator
priveleges on you computer then see Section~\ref{sec:config-build} for 
information on how to specify an alternative installation directory.

Building the Vision Workbench under Windows is possible, but it is not
currently automatically supported.  The easiest thing to do is to
simply include the \verb#.cc# files from the Vision Workbench modules
that you want to use directly in your own project file.  You will of
course still need to install the Boost libraries as well as any other
libraries you want to use.  Pre-built Windows versions of a number of
libraries, such as the JPEG, PNG, and TIFF libraries, are available
online from the GnuWin32 project at \verb#gnuwin32.sourceforge.net#.  
You will of course need to configure your project's include file and 
library search paths appropriately.  Also be sure to configure your 
project to define the preprocessor symbol NOMINMAX to disable the 
non-portable Windows definitions of \verb#min()# and \verb#max()# 
macros which interfere with the standard C++ library functions of 
the same names.

\section{A Trivial Example Program}

Now that you've built and installed the Vision Workbench let's start
off with a simple but fully-functional example program to test things
out.  The full source code is shown in Listing \ref{lst:vwconvert.cc}.
You should be able to obtain an electronic copy of this source file
(as well as all the others listed in this book) from wherever you
obtained this document.  For now don't worry about how this program
works, though we hope it is fairly self-explanatory.  Instead, just
make sure that you can build and run it successfully.  This will
ensure that you have installed the Vision Workbench properly on your
computer and that you have correctly configured your programming
environment to use it.

\sourcelst{vwconvert.cc}{A simple demonstration program that can
copy image files and convert them from one file format to another.}

The program reads in an image from a source file on disk and writes it 
back out to a destination file, possibly using a different file format.  
It takes the source and destination filenames as two command-line arguments.  
For example, to convert a JPEG image called \verb#image.jpg# in the current 
directory into a PNG image you might say:
\begin{verbatim}
  vwconvert image.jpg image.png
\end{verbatim}
The details of how you invoke a command-line tool like this will of 
course vary from one operating system to another.  Note that exactly 
what image file formats are support will depend on what file format 
libraries you have installed on your system.

\section{Configuring the Build System}\label{sec:config-build}

The Vision Workbench build system offers a variety of configuration
options which you provide either as command-line flags to the 
\verb#configure# script.  We'll discuss a few of the most important
options here, but for a more complete list you can run ``\verb#./configure --help#''.  
As an alternative to specifying command-line flags every time, you may
instead create a file called \verb#config.options# with your preferences 
in the base directory of the Vision Workbench repository.  A file
called \verb#config.options.example# is provided that you can copy and
edit to your liking.  Note that none of this has any impact on Visual
Studio users, who must instead configure their projects by hand.

The single most important option is probably the \verb#--with-paths=PATHS# 
option, where you replace \verb#PATHS# with a whitespace-separated list of 
paths that the build system should search when looking for installed 
libraries.  For example if you specify the option \verb#--with-paths=/foo/bar# 
then it will search for header files in \verb#/foo/bar/include#, library 
files in \verb#/foo/bar/lib#, and so on.  The default search path includes 
a number of common locations for user-installed libraries, such as 
\verb#/usr/local#, \verb#$(HOME)/local#, and \verb#/sw#.  The \verb#PKG_PATHS# 
configuration file variable has the same effect as this option.

The next most important options have the form \verb#--enable-module-foo[=no]#, 
where \verb#foo# is replaced by the lower-case name of a module such
as \verb#mosaic# or \verb#gpu#.  Unsurprisingly, this allows you to
control whether or not certain modules are built.  Disabling modules
that you do not use can speed up compilation and testing time, which
is especially useful if you are making changes to the Vision Workbench
source and need to recompile often.  The corresponding configuration
file variables have the form \verb#ENABLE_MODULE_FOO#, in all-caps,
and are set to either \verb#yes# or \verb#no#.

Two handy options, \verb#--enable-optimize# and \verb#--enable-debug#,
determine the compiler options used when building the few library
files.  You can again specify an optional argument of the form
\verb#=no# to disable the corresponding feature, and you can also
specify a particular optimization level in the same manner.  For
example, if you want to make it as easy as possible to debug Vision 
Workbench code using a debugger you might say 
\verb#--enable-optimize=no --enable-debug# 
to disable all optimizations and include debugging symbols.  The corresponding
configuration file variables are \verb#ENABLE_OPTIMIZE# and
\verb#ENABLE_DEUBG#.  Keep in mind that since most Vision Workbench
code is header-only you should remember to configure your own project
similarly or you may not notice any difference.

Finally, to specify that the build system should install the Vision Workbench 
someplace other than \verb#/usr/local#, specify the path using the 
\verb#--prefix=PATH# option.   The corresponding configuration file 
variable is, of course, called \verb#PREFIX#.
