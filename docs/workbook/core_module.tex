\chapter{{\tt Core} Module}
\label{ch:core-module}

The Core Module contains fundamental tools for building good software infrastructure.

 \section{Vision Workbench Exceptions}
\label{sec:exceptions}

The Vision Workbench is intended in part to be used in flight systems,
experimental multiprocessor systems, or other environments where
exceptions may not be fully supported.  As a result, the use of
exceptions within the Vision Workbench is tightly controlled.  In
particular, the exception usage rules were designed to minimize the
impact on platforms that do not support exceptions at all.  There is a
standard Vision Workbench "exception" class hierarchy which is used to
describe errors and can be used even on platforms that do not support
the C++ exception system.

The \verb#vw::Exception# class serves as a base class for all VWB error
types.  It is designed to make it easy to throw exceptions with
meaningful error messages.  For example, this code:

\begin{verbatim}
  vw_throw( vw::Exception() << "Unable to open file \"" << filename << "\"!" );
\end{verbatim}

would generate a message like this:

\begin{verbatim}
  terminate called after throwing an instance of 'vw::Exception'
       what():  Unable to open file "somefile.foo"!
\end{verbatim}

Note that in the example the exception was thrown by calling the
\verb#vw_throw()# function rather than by using the C++ throw
statement.  On platforms that do support C++ exceptions the default
behavior for \verb#vw_throw()# is to throw the exception in the usual
way.  However, the user can provide their own error-handling mechanism
if they choose.  For example, the default behavior when exceptions are
disabled is to print the error text to \verb#stderr# and call
\verb#abort()#.

There are a number of standard exception types that derive from
\verb#vw::Exception#.  These are shown in Table
\ref{tbl:exception-types}.  In the above example, the exception should
probably have been of type \verb#vw::IOErr#.

Also, two macros, \verb#VW_ASSERT(condition,exception)# and
\verb#VW_DEBUG_ASSERT(condition,exception)#, are provided, with the
usual assertion semantics.  The only difference is that the debug
assertions will be disabled for increased performance in release
builds when \verb#VW_DEBUG_LEVEL# is defined to zero (which happens by
default when \verb#NDEBUG# is defined).

\begin{table}[t]\begin{centering}
\begin{tabular}{|l|l|l|} \hline
Function & Description \\ \hline \hline
\verb#ArgumentErr# & Invalid function argument exception \\ \hline
\verb#LogicErr# & Incorrect program logic exception \\ \hline
\verb#InputErr# & Invalid program input exception \\ \hline
\verb#IOErr# & IO (usually disk IO) failure exception \\ \hline
\verb#MathErr# & Arithmetic failure exception \\ \hline
\verb#NullPtrErr# & Unexpected NULL pointer exception\\ \hline
\verb#TypeErr# & Invalid type exception \\ \hline
\verb#NotFoundErr# & Not found exception \\ \hline
\verb#NoImplErr# & Unimplemented functionality exception \\ \hline
\verb#Aborted# & Operation aborted partway through \\ \hline

\end{tabular}
\caption{Vision Workbench exception types that derive from
  vw::Exception.  All behave like C++ output stream classes, so you
  can associate an error message with the exception using the
  stream operator.}
\label{tbl:exception-types}
\end{centering}\end{table}

Exceptions are enabled or disabled based on the value of the
\verb#VW_ENABLE_EXCEPTIONS# macro defined in \verb#vw/config.h#.  This
value can be set by passing the command line options
\verb#--enable-exceptions# (the default) or \verb#--disable-exceptions#
to the configure script prior to building the Vision Workbench.  This
option also sets an automake variable called \verb#ENABLE_EXCEPTIONS#
which may be used by the build system to conditionally compile entire
source files.

In either case the default behavior of \verb#vw_throw()# may be
overridden by passing a pointer to a user-defined object derived from
\verb#ExceptionHandler# to \verb#vw::set_exception_handler()#.  The
user specifies the error-handling behavior by overriding the abstract
method \verb#handle()#.

\section{The System Cache}
\label{sec:cache}

The Vision Workbench provides a thread-safe system for caching
regeneratable data.  When the cache is full, the least recently used
object is {\em invalidated} to make room for new objects.  Invalidated
objects have had the resource associated with them (e.g. memory or
other resources) deallocated or freed, however, the object can be {\em
  regenerated} (that is, the resource is regenerated automatically by
the cache) when the object is next accessed.

The \verb#vw::Cache# object defined in \verb#src/vw/Core/Cache.h# can
be used to store any resource.  For example, one common usage would be
to create a cache of image blocks in memory.  In this case, the cache
enforces a maximum memory footprint for image block storage, and it
regenerates the blocks (e.g. reloads them from a file on disk) when
necessary if a block is accessed.

One can also cache more abstract resource types, such as std::ofstream
objects pointing at open files on disk. The following section
describes this use case in detail.

\subsection{Example: Caching std::ofstream}

Consider a situation wherein your system needs to open and read from
tens of thousands of tiny files on disk.  There is a high degree of
locality of access, meaning that once you start reading from a file,
you are likely to read from it again in the near future, but that
likelihood diminishes as time goes on.  It is impractical to simply
open all of the files at once (this would eat up memory in your
program, plus there is a hard limit in most operating systems on the
number of open files you can have at any one time).  However, it could
potentially be very slow to close and re-open the file each time you
attempt to read from it because of the time it takes to parse the file
header or seek to the correct location.

This situation calls for a cache.

We begin by specifying a {\em generator} class whose sole purpose is
to regenerate the resource if it is ever necessary to reopen it.  In
this case, our class contains a \verb#generate()# method that return a
shared pointer to a newly open ifstream.

\begin{verbatim}
  class FileHandleGenerator {
    std::string m_filename;
  public:
    typedef std::ifstream value_type;

    FileHandleGenerator( std::string filename ) : m_filename( filename ) {}

    // The size is useful when managing items that have
    // a known size (e.g. allocated blocks in memory).
    // When caching abstract items, like open files, the
    // size does not matter, only the total number of open
    // files, hence the size in this case is 1.
    size_t size() const { return 1; }

    // Generate is called whenever there is a cache miss.  In
    // this example, we reopen the file in append mode.
    boost::shared_ptr<value_type> generate() const {
      return boost::shared_ptr<value_type> ( new std::ifstream(m_filename, ios::app) );
    }
  };
\end{verbatim}

Next, we create a \verb#vw::Cache# object for storing instances of our
\verb#FileHandleGenerator# class.  The cache itself can be declared
without knowing the type of the generator(s) that will be inserted
into it.  The \verb#vw::Cache# constructor takes only one argument
specifying the size of the cache.  This value will be used in
conjunction with the \verb#size()# methods its cache generators to
determine when the cache is full.  If the sum of the \verb#size()#
values from {\em valid} generators exceeds the max cache size, the
least recently used cache generator is invalidated to make room for a
new call to \verb#generate()#.

\begin{verbatim}
  // This cache maintains up to 200 open files at a time.
  static vw::Cache filehandle_cache( 200 );

  // Insert a ifstream generator into the cache and store a handle
  // so that we can access it later.
  Cache::Handle<FileHandleGenerator> file_ptr;
  file_ptr = filehandle_cache().insert(GdalDatasetGenerator(filename));

  // ... time passes ...

  // We access the file generator like a pointer.  The cache
  // will re-open the file if necessary.
  char[2048] line_from_file;
  *file_ptr.get_line(line_from_file, 2048);
\end{verbatim}

Note in this example that our call to \verb#insert()# returns a
\verb#Cache::Handle<># object that ``points at'' the cache's version
of the ifstream object.  The cache handle behaves like a pointer, and
we can use it just as we would a normal C++ pointer to a
\verb#std::ifstrem#.  However, if the cache closes the file at any
point to make room for other files, this cache handle ensures that the
file is regenerated (re-opened) the next time we try to access it.

\subsection{Performance Considerations and Debugging}

Use of a cache can greatly increase the efficiency of a program by
storing information that would be expensive to regenerate for as long
as possible.  However, depending on the size of the cache and the
pattern you are using to access it, one can inadvertently end up in a
situation where performance may suffer considerably.

For example, consider a scenario where you have a
\verb#DiskImageView<># that points at a very large (22,000x22,000
pixels) 8-bit RGB image that is stored in a file format with a block
size of 2048x2048 pixels.  The \verb#DiskImageView<># caches these
block as the image is accessed, and each block takes ~96-MB of memory.

If we access the \verb#DiskImageView<># one scan line at a time (during
a rasterization operation, for example), the cache will need to store
at least $22,000/2048 = 10$ blocks in memory for efficient access (i.e
our cache size must be >= 960-MB).  If the cache is too small, the
left-most image blocks in the row will be invalidated by the cache to
make room for the right-most blocks, and vice versa as we traverse
{\em each scanline} of the \verb#DiskImageView<>#.  Given that it
takes a few seconds to regenerate each block, this would be hugely
inefficient.

If your code is running much, much slower than you expect, you may
have a similar cache miss problem.  You can debug this by observing
the ``cache'' {\em log namespace} using the system log (see Section
\ref{sec:log} for details).  The cache subsystem logs debugging
information whenever cache handles are invalidated or regenerated.
Use these messages to get a gross idea of how the cache is
performing.

\section{The System Log}
\label{sec:log}

As the Vision Workbench has become more parallelized, and as new
subsystems (e.g. caching, fileio, threadpool management, etc.) have
been added, it has become increasingly challenging to monitor and
debug the code base.  The Vision Workbench log class was designed to
address the evolving needs of VW developers and users alike.  The
following design guidelines summarize the features of the system log
facility provided in \verb#vw/Core/Log.h#.

\begin{itemize}
\item {\bf Thread Safety}: Log messages are buffered on a
  per-thread basis so that messages form different threads are
  correctly interleaved one line at a time in the log output.
\item {\bf Log Granularity}: Users will want to monitor different
  subsystems at different times.  Each log message includes an
  associated {\em log namespace} and {\em log level}.  Used in
  conjunction with the the \verb#LogRuleSet# class, the log namespace
  and level allow the user to monitor only the log messages that
  concern them
\item {\bf Multiple Log Streams}: Log messages can be directed either
  to the user's terminal, one or more files on disk, or both at the
  same time.  Each log stream has its own \verb#LogRuleSet#, so it is
  possible to tailor the log output that appears in each file.
\item {\bf Runtime Log Adjustment}: When something goes wrong in the
  middle of a long job, it is undesirable to stop the program to
  adjust the log settings.  The VW log framework will reload the log
  settings from the \verb#~/.vwrc# file (if it exists) every
  5 seconds, so it is possible to adjust log settings during program
  execution.  See Section \ref{sec:logconf-file} below.
\end{itemize}

\subsection{Writing Log Messages}

Logging in the Vision Workbench is simple.  Just call the
\verb#vw_out(log_level, log_namespace)# command, which will return a
\verb#basic_ostream# object that you can stream into using the standard C++
\verb#<<# operator.  For example,

\begin{verbatim}
// Record the default number of Vision Workbench threads to the log
int num_threads = vw::Thread::default_num_threads();
vw_out(vw::InfoMessage, "thread") << "The default number of threads is " << num_threads << "\n";
\end{verbatim}

This would generate a log message that looks something like this.

\begin{verbatim}
2007-Dec-28 14:31:52 {0} [ thread ] : The default number of threads is 8.
\end{verbatim}

The log message includes an {\em infostamp} consisting of the timestamp,
the unique id of the thread that generated the log message, and the
\verb#log_namespace# specified as an argument to \verb#vw_out#.

\begin{verbatim}
<date> <time> {<thread_id>} [ <log_namespace> ] : <log_message>
\end{verbatim}

Note that, for aesthetic reasons, log messages that go to the console
only print the \verb#log_message# after the colon; the {\em infostamp} is
omitted. A new {\em infostamp} is prepended to the log stream each
time \verb#vw_out()# is called.

Take note of the newline character at the end of stream to
\verb#vw_out()# in the example above.  The logging framework will
cache the log message until it sees the newline {\em at the end} of a
call to \verb#operator<<#, at which point the log stream is flushed,
and newline is written, starting a new line in the log file.
Therefore, it is highly recommended that you end every log message
with a newline character (or \verb#std::endl#).

\subsection{Progress Bars}

Progress Bars are also a part of Vision Workbench logging capabilities
and they are handled with Progress Callbacks. Currently the only
implementation is the \verb#TerminalProgressCallback# and it shares a
lot in common with \verb#vw_out#. Here's an example:

\begin{verbatim}
TerminalProgressCallback bar("image", "Writing:");
bar.report_progress(0);
// When inside process:
bar.report_progress(.5); // Display percentage 50
// When outside process:
bar.report_finished();
\end{verbatim}

The first argument is the log namespace which is the same as
before. The second argument is the text that goes in front of the
progress bar. The above code would produce the following output in the
console:

\begin{verbatim}
Writing:[********************************.................................] 50%
\end{verbatim}

Then when it's finished it will look like:

\begin{verbatim}
Writing:[************************************************************] Complete!
\end{verbatim}

Notice that the progress bars, even with the extra text, do not exceed
80 characters. Added text before the progress bar that are too long
and don't allow for the bar to fit inside 80 characters will throw an
\verb#ArgumentErr()# exception.

\subsection{The Log Configuration File}
\label{sec:logconf-file}

The {\em log configuration file} can be used to change the Vision
Workbench logging behavior in real-time, even while your program is
running.

Every five seconds, or when a log message is generated using
\verb#vw_out()# (whichever is {\em longer}, the system log checks to
see if the logconf file has been modified.  If so, it erases all log
streams and log rules, and reloads these settings from the file.  If
you modify the file while your program is running, you will see your
changes take affect anywhere from 0-5 seconds from the time that you
save your changes.

Using the file, you have full control over the system log: you can
create as many log streams to files as you like, and you can adjust
the log rules for both the console log and the log file streams.  Note
that syntax errors and malformed statements are silently ignored by
the log configuration file parser, so check your file carefully for
errors prior to saving.

An example log configuration file appears in
Listing~\ref{lst:LogConf.example}.

\sourcelst{LogConf.example}{An example log configuration file.}

Note also in the example log configuration the line that shows how to
silence all progress bars. This is helpful for headless server system
who don't appreciate progress bars. Progress bars secretly append
\verb#.progress# to their log's namespace. So for example: If you have
a progress bar in namespace \verb#kitty#. You can silence it in the
log configuration file with \verb#0 = kitty.progress#.

\subsection{System Log API}
If you would rather not use the log configuration file, you can adjust
the system log settings directly by accessing the singleton instance of
the vw::Log class using \verb#vw::system_log()#.  Once you have done
this, you can explicitly add (or clear) new log streams (or
\verb#LogInstance# objects, in the parlance of the API), and you can
adjust log rule sets (using the \verb#LogRuleSet#) class.

Consult the API documentation for \verb#vw/Core/Log.h# for more
information.

