\chapter{The Cartography Module}\label{ch:cartography-module}

[{\em Note: The documentation for this chapter is not yet complete... ]}
$$
$$ 
The earliest robots in space were not planetary rovers -- they were
unmanned probes that studied the planetary bodies in our solar system
from afar.  Today there are roughly twenty extraterrestrial spacecraft
in active communication with earth (and only two planetary rovers), so
the bulk of the extraterrestrial data that we receive consists of
imagery that originated on round(ish) surfaces.  The natural thing to
do with this data is to merge it together into a map, but when doing
so we are faced with the same problem that has plagued cartographers
for hundreds of years: how does one flatten the globe?  This is the job
of the Vision Workbench Cartography module: to make maps.

Before diving into an introduction on planetary cartography, we will
point out another problem that is relatively new to Cartography.  The
amount of map data that we have collected about Earth and the other
bodies in our solar system is {\em immense}.  It is often impossible
to store an entire mapping data set in memory all at once, so
intelligent paging, caching, and storage strategies must be used in
order to make working with this data tractable.  For this reason, the
Cartography module is particularly powerful when used in conjunction
with Mosaic module (see Chapter \ref{ch:mosaic-module}), which is
designed to efficiently process and combine extremely large data sets.

We will begin this chapter with a quick summary of the third party
libraries that are needed to compile the Vision Workbench.  We will
then describe the \verb#GeoRefence# class, which creates a
relationship between pixel coordinates in an image and coordinates on
a globe.  Next, we will discuss the \verb#GeoTransform# class, which
provides a simple means of re-projecting map data.  We finish this
section with methods for reading and writing image files with embedded
geospatial metadata.

\section{Software Dependencies}

The Cartography module is currently built on top of two third party
libraries:

\begin{itemize}
\item GDAL   [ {\tt http://www.remotesensing.org/gdal/ } ]
\item Proj.4 [ {\tt http://proj.maptools.org/ }]
\end{itemize}

In order to enable the Cartography module, you must have these
libraries installed on your system before you configure and build the
Vision Workbench.  You may need to use the \verb#PKG_PATHS# directive
in the \verb#config.options# file if you install them in a
non-standard location as discussed in Chapter \ref{ch:gettingstarted}.

Once the library for the Cartography module has been built, the header
files for GDAL and Proj.4 are no longer needed, so you can rely solely
on linking in libraries when building your own application.

\section{The {\tt GeoReference} Class}

When you point at a location on a map, you probably want to know where
that location can be found in the real world.  This relationship
depends first and foremost on the familiar notion of a map's scale.
However, this relationship is also affected by a subtle, but extremely
important dependence on how the map is {\em projected}.  That is, the
image depicts a scene that sits on the surface of a spheroid.
However, the image is flat, so at best it represents a very slightly
distorted view of the surface.  

One can imagine all sorts of different ways that the surface can be
warped or projected onto a flat plane (or, at the very least,
projecting onto a manifold that can be unfolded into a plane without
distorting distances and areas -- a sphere cannot be unfolded in this
way).  Generations of cartographers have struggled with this
topological challenge, and as a result they have developed many
different ways to ``un-fold'' the globe so that it can be represented
as a flat image.  Rather than attempt a description of these many
techniques here, we suggest you look at this excellent web site
describing all aspects of map projections.

\begin{verbatim}
  http://www.progonos.com/furuti/MapProj/CartIndex/cartIndex.html
\end{verbatim}

The Proj.4 manual is also recommended as a reference for the specific
map projections supported by the Vision Workbench.

Now would be a good time to take a break from reading this section of
the documentation to look over these references.  When you return, we
will dive into some code examples.  

\subsection{The Datum}

A Vision Workbench \verb#GeoReference# object is composed of three items:

\begin{itemize}
\item {\bf The Projection}: As discussed above, this is the technique
  used to represent the round globe in a flat image.
\item {\bf The Affine Transform}: This is the geometric transformation
  between pixel coordinates in the image to coordinates in the map
  projection space.
\item {\bf The Datum}: Describes the approximate shape of the
  planetary body, as either a sphere or an ellipsoid.
\end{itemize}

\subsection{The Affine Transform}

Let's start by being explicit about the coordinate systems we will be
working with.  For images, we adopt the usual Vision Workbench
coordinate system wherein the upper left corner of the image is the
origin, the $u$ coordinate increases as you move right along the
columns of the image, and the $v$ coordinate increases as you move
down the rows.  

For a planetary body, the coordinate of a point on the surface is
typically measured in latitude, longitude, and radius ($\phi, \theta,
r$).  Lines of latitude are perpendicular to the axis of rotation and
are measured from the center line, the {\em equator} (+/-90 degrees).
Lines of Longitude are vertical, passing through both the North and
South poles of the planet.  It is measured from a vertical arc on the
surface called the {\em meridian}.  We will generally adopt an East
positive frame of reference (latitude increases to the east of the
meridian 0-360 degrees).  Finally, the radius is measured from the
point to the planet's center of mass.  Note that this coordinate
system is similar but not identical to spherical coordinates in a
mathematical sense, where ``latitude'' would be measured from the
North pole rather than the equator.

Under this set of assumptions, if we have a point $P_{img} = (u,v)$ in
the image, and we want to relate it to some planetary coordinates

\subsection{Putting Things Together}
%% - The GeoReference Object
%%   - Creating 
%%     - proj4str
%%     - datum and transform
%%   - << streaming
%%   - Setting projections

\begin{table}[t]\begin{centering}
\begin{tabular}{|l|l|l|} \hline
Method & Description \\ \hline \hline
\verb#set_sinusoidal()# & Sinusoidal Projection \\ \hline
\verb#set_mercator()# & Mercator Projection \\ \hline
\verb#set_orthographic()# & Orthographic Projection \\ \hline
\verb#set_stereographic()# & Stereographic Projection \\ \hline
\verb#set_UTM()# & Universal Transverse Mercator (UTM) Projection (Earth only) \\ \hline
\end{tabular}
\caption{Currently supported {\tt GeoReference} map projections.}
\label{tbl:georeference-map-projections}
\end{centering}\end{table}


\section{Geospatial Image Processing}
\subsection{The {\tt GeoTransform} Functor}
%%     - Use to reproject maps 
%%     - Formulate a source and destination \verb#GeoReference# object
%%     - 
%%   - \verb#xyz_to_latlon()#

\section{Georeferenced File I/O}
\subsection{{\tt DiskImageResourceGDAL}}
